\documentclass[parskip=half-]{scrartcl}

\usepackage[margin=1in]{geometry}
\pagestyle{empty}
\usepackage{graphicx}
\usepackage{newtxtext}
\usepackage{array}

%\renewcommand{\familydefault}{\sfdefault}

\title{Gender Inclusive Language in English}
\subtitle{Simple Version}
\author{Pao Inclusiveness Working Group}
\date{}

\renewcommand{\arraystretch}{1.2}

\begin{document}
\maketitle
\thispagestyle{empty}

Gender-inclusive language acknowledges and respects \textbf{all} gender identities. It additionally promotes equality and helps reduce harmful biases and stereotypes.

\begin{itemize}
	\item Use vocabulary inclusive of all genders when possible.

		\begin{tabular}{|c|c|}
			\hline
			Avoid & Use
			\tabularnewline\hline
			Chairman & Chair/Chairperson
			\tabularnewline\hline
			Girls and boys & Students, etc.
			\tabularnewline\hline
			Sisters and brothers & Siblings
			\tabularnewline\hline
			\{Boy, Girl\}-friend & Loved one
			\tabularnewline\hline
		\end{tabular}

		In general, avoid making gender assumptions.
	
	\item Respect people's pronouns. Ask for pronouns if unsure. Include your own pronouns when introducing yourself.

		The following is a table of common pronouns.

		\begin{tabular}{|c|c|c|c|c|c|}
			\hline
			Association & Nominative & Accusative & Prenomial P. & Postnomial P. & Reflexive
			\tabularnewline\hline
			Inclusive & They & Them & Their & Theirs & Themself
			\tabularnewline\hline
			Feminine & She & Her & Her & Hers & Herself
			\tabularnewline\hline
			Masculine & He & Him & His & His & Himself
			\tabularnewline\hline
		\end{tabular}

		Note that the ``Association'' column represents traditional gender association of the pronoun set. People who use particular pronouns may or may not fall into these traditionally-associated genders.

		Specially, ``they/them'' is not a gender association, it may be used generically. The use of ``they/them'' when referring to a singular person is quite common in contemperory English.

		When referring to a placeholder entity (e.g. ``someone left their laptop here'') or a person whose pronouns you do not know, either avoid pronouns altogether, or use gender-inclusive pronouns.
	
	\item The bottom line.

		Do not use words that represent LGBTQIA+ identities, or for that matter, any word that represents a minority or other group, in insulting or derogatory ways. Saying that something is ``gay'' or ``homosexual'' to express hatred towards said entity shows profound disrespect for minorities that have been historically disadvantaged.  The argument to ``take this easy as these are just jokes'' is ridiculous. The mere act of descriptive terms for a specific social group being used as a ``joke'' undermines the seriousness of the social issue---it is worth finding something else to joke about.
\end{itemize}

\end{document}
